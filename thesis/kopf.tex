%*************************************************************
% Master Project                                             *
% Ing. Minerva Gabriela Vargas Gleason                       *
% IAI - Institute of Artificial Intelligence                 *
% Universität Bremen                                         *
%                                                            *
% pdfLaTex                                                   *
%*************************************************************



%%%%%%%%%%%%%%%%%%%%%%%%%%%%%%%%%%
%% Optionen zum Layout          %%
%%%%%%%%%%%%%%%%%%%%%%%%%%%%%%%%%%%

%% Dokumentenklasse (Koma Script) -----------------------------------------
\documentclass[%
   %draft,     % Entwurfsstadium
   final,      % fertiges Dokument
%%%% --- Schriftgr��e ---
   12pt,
   %smallheadings,    % kleine �berschriften
   %normalheadings,   % normale �berschriften
   %bigheadings,       % gro�e �berschriften
%%%% === Seitengr��e ===
   % letterpaper,
   % legalpaper,
   % executivepaper,
   a4paper,
   % a5paper,
   % landscap,
%%%% === Optionen f�r den Satzspiegel ===
   %BCOR5mm,          % Zusaetzlicher Rand auf der Innenseite
   %DIV11,            % Seitengroesse (siehe Koma Skript Dokumentation !)
   %DIVcalc,         % automatische Berechnung einer guten Zeilenlaenge
   %1.1headlines,     % Zeilenanzahl der Kopfzeilen
   %headinclude,     % Kopf einbeziehen
   %headexclude,      % Kopf nicht einbeziehen
   %footinclude,     % Fuss einbeziehen
   %footexclude,      % Fuss nicht einbeziehen
   %mpinclude,       % Margin einbeziehen
   %mpexclude,        % Margin nicht einbeziehen
   pagesize,         % Schreibt die Papiergroesse in die Datei.
                     % Wichtig fuer Konvertierungen
%%%% === Layout ===
   oneside,         % einseitiges Layout
   %twoside,          % Seitenraender f�r zweiseitiges Layout
   %onecolumn,        % Einspaltig
   %twocolumn,       % Zweispaltig
   %openany,         % Kapitel beginnen auf jeder Seite
   openright,        % Kapitel beginnen immer auf der rechten Seite
                     % (macht nur bei 'twoside' Sinn)
   %cleardoubleplain,    % leere, linke Seite mit Seitenstil 'plain' 
   cleardoublepage=empty,% leere, linke Seite mit Seitenstil 'empty'
   titlepage,        % Titel als einzelne Seite ('titlepage' Umgebung)
   %notitlepage,     % Titel in Seite integriert
   captions=tableheading,
%%%% --- Absatzeinzug ---
   %                 % Absatzabstand: Einzeilig,
   %parskip,         % Freiraum in letzter Zeile: 1em
   %parskip*,        % Freiraum in letzter Zeile: Viertel einer Zeile
   %parskip+,        % Freiraum in letzter Zeile: Drittel einer Zeile
   %parskip-,        % Freiraum in letzter Zeile: keine Vorkehrungen
   %                 % Absatzabstand: Halbzeilig
   %halfparskip,     % Freiraum in letzter Zeile: 1em
   %halfparskip*,    % Freiraum in letzter Zeile: Viertel einer Zeile
   %halfparskip+,    % Freiraum in letzter Zeile: Drittel einer Zeile
   %halfparskip,     % Freiraum in letzter Zeile: keine Vorkehrungen
   %                 % Absatzabstand: keiner
   parskip=false,    % Einger�ckt (Standard)
%%%% --- Kolumnentitel ---
   headsepline,      % Linie unter Kopfzeile
   %headnosepline,   % keine Linie unter Kolumnentitel
   footsepline,      % Linie �ber Fusszeile
   %footnosepline,   % keine Linie unter Fussnote
   plainfootsepline, % Linie �ber Fu�zeile auf "`leeren Seiten"'
%%%% --- Kapitel ---
   %chapterprefix,   % Ausgabe von 'Kapitel:'
   %nochapterprefix,  % keine Ausgabe von 'Kapitel:'
%%%% === Verzeichnisse (TOC, LOF, LOT, BIB) ===
   %listof=totoc,      % Tabellen & Abbildungsverzeichnis ins Inhaltsverzeichnis
   listof=notoc,	  % Tabellen & Abbildungsverzeichnis ins nicht ins Inhaltsverzeichnis
   %idxtotoc,        % Index ins TOC
   bibliography=totoc, % Bibliographie ins TOC
   %bibtotocnumbered, % Bibliographie im TOC nummeriert
   %liststotocnumbered, % Alle Verzeichnisse im TOC nummeriert      
   toc=graduated,    % eingereuckte Gliederung
   %tocleft,         % Tabellenartige TOC
   listof=graduated,      % eingereuckte LOT, LOF
   %listsleft,       % Tabellenartige LOT, LOF
   %pointednumbers,  % �berschriftnummerierung mit Punkt, siehe DUDEN !
   numbers=noenddot, % �berschriftnummerierung ohne Punkt, siehe DUDEN !
   %openbib,         % alternative Formatierung des Literaturverzeichnisses
%%%% === Matheformeln ===
   %leqno,           % Formelnummern links
   %fleqn,            % Formeln werden linksbuendig angezeigt
]{scrreprt}%     Klassen: scrartcl, scrreprt, scrbook
% -------------------------------------------------------------------------

\usepackage{natbib} %For library
\renewcommand{\bibsection}{\chapter*{References}}
\usepackage{bm} % To write italic-bold text in formulas
%\usepackage{subcaption} % Place figures side-by-side
\usepackage{graphicx}

\usepackage[T1]{fontenc}

\usepackage[latin1]{inputenc}
\usepackage[english]{babel}
\usepackage[automark]{scrpage2}
\usepackage{epsfig}
\usepackage{graphicx}  %F�r Grafiken
\usepackage[cmex10]{amsmath}
\usepackage{amssymb,amsfonts}
\usepackage{mathptmx} %Sch�ne nicht Pixelige Times-Schrift
%\usepackage{amsmath,amssymb,amsfonts}  % Wichtige Matheumgebungen
\usepackage{fancybox}
\usepackage{setspace}

%--- F�r TPX -----
\usepackage{ifpdf}
\usepackage{pgf}
%-----------------

\usepackage{scrhack} %f�r das float-package im Komascript
\usepackage{float} % f�r die Positionsangabe [H]	HERE!: hier, nirgendwo sonst...

% Das Eindrücken am Absatzbeginn verhindern:
\setlength{\parindent}{0pt}


\usepackage{upgreek} % f�r nicht-kursiv griechische Symbole
% \upphi = phi

%% CODE FORMAT
\usepackage{listings} % Fur Programmcodeumgebung \
\renewcommand\lstlistingname{Source Code}   % Neue Bezeichnung fur Listing

\usepackage{color}
\definecolor{codegreen}{rgb}{0,0.6,0}
\definecolor{codegray}{rgb}{0.5,0.5,0.5}
\definecolor{codepurple}{rgb}{0.58,0,0.82}
\definecolor{backcolour}{rgb}{0.95,0.95,0.92}
\lstdefinestyle{mystyle}{
	backgroundcolor=\color{backcolour},   
	commentstyle=\color{codegreen},
	keywordstyle=\color{magenta},
	numberstyle=\tiny\color{codegray},
	stringstyle=\color{codepurple},
	basicstyle=\footnotesize,
	breakatwhitespace=false,         
	breaklines=true,                 
	captionpos=b,                    
	keepspaces=true,                 
	numbers=left,                    
	numbersep=5pt,                  
	showspaces=false,                
	showstringspaces=false,
	showtabs=false,                  
	tabsize=2
}

\lstset{style=mystyle}
%% Finish code format

% Das Hyperref Packet !!!
\usepackage[
		pdftex,
		pdftoolbar=true, hyperfootnotes=false,breaklinks=true,
		pdfpagelabels=true,
		bookmarks, bookmarksopen, bookmarksnumbered, bookmarksopenlevel=1,
		pdfauthor={Minerva Vargas},
		pdfsubject={Master Project},%
		pdfkeywords={Bachelorarbeit, Innere-Punkte-Verfahren, Simplex-Algorithmus, Unterbestimmte Gleichungssysteme, Compressed Sensing,...},
		pdftitle={Master Project, Minerva Vargas},
		pdfstartview={FitH},
		pdfborder={0 0 0},  %Farbe aller Linkumrandung auf Wei� gesetzt
		plainpages=false]
{hyperref}


\usepackage{tabularx} %erstellt Zeilenumbruch in Tabellen bestimmter Gr��e: %\begin{tabularx}{\textwidth}{|c|c|X|c|} ...\end{tabularx}
\usepackage{booktabs} %F�r dicke Linien in Tabellen, z.B. \toprule, \midrule, \cmidrule(lr){4-5} oder \bottomrule

% Pseudo-Code:
\usepackage{algorithm}
\usepackage{algpseudocode}
\renewcommand{\algorithmicrequire}{\textbf{Input:}}
\renewcommand{\algorithmicensure}{\textbf{Output:}}

% TODO NOTES
\usepackage[colorinlistoftodos,prependcaption,textsize=tiny]{todonotes}

% Schriftfamilie aller Gliederungs�berschriften �ndern
\setkomafont{sectioning}{\rmfamily}
\setcapindent{1em} % - Teilweise h�ngender Einzug bei Beschriftung von Abbildungen und Tabellen
%\setkomafont{captionlabel}{\bfseries} % Tabellen & Abbildung Dick geschrieben
%Legt die Einr�cktiefe der ersten Zeile f�r alle folgenden Abs�tze fest.
\setlength{\parindent}{0cm}

% Der Pfad zu den Grafiken, relativ zum Ordner wo das LaTex-Dokument liegt
\graphicspath{{bilder/}}

\setcounter{secnumdepth}{4} % Subsubsections nummerieren!!!
\setcounter{tocdepth}{4}    % Subsubsections mit ins Inhaltsverzeichnis!!!


%Seitenlayout f�r die Seiten auf denen ein neues Kapitel beginnt
\usepackage{titlesec}
\usepackage{sectsty}
%--> Kapitel + Nummer + Trennlinie + Name + Trennlinie
\titleformat{\chapter}[display]	% {command}[shape]
  {\usekomafont{chapter}\Huge}	% format
  {   										% label
  \normalsize\MakeUppercase{\chaptertitlename} \Large \thechapter %\filright%
  }%}
  {1pt}										% sep (from chapternumber)
  {\titlerule \vspace{0.0pc} \filright }   % {before}[after] (before chaptertitle and after)
  [\vspace{0.8pt} \filright {\titlerule}]

%Seitenlayout
\setlength{\textwidth}{16cm}
\setlength{\textheight}{23cm}
\setlength{\oddsidemargin}{0.2cm}
\setlength{\evensidemargin}{0.2cm}
\addtolength{\topmargin}{-1cm}
\renewcommand{\baselinestretch}{2}
\setlength{\parskip}{1.3ex}
\setlength{\headheight}{1.4\baselineskip}

\pagestyle{scrheadings} 
\ihead{}
\chead{}
\ohead{\rmfamily \small \headmark} 
\ofoot[\rmfamily \small \pagemark]{\rmfamily \small \pagemark}
\cfoot[]{}
\ifoot[\\ \vspace{0.2cm} \psfig{figure=./ai_logo.png,height=1.1cm}]{\\ \psfig{figure=./ai_logo.png,height=1.1cm}} 


%\usepackage{tikz}
%\usetikzlibrary{calc}
%\tikzset{offset/.style={to path={%
%    -- ($(\tikztostart)!#1cm!(\tikztotarget)$)}},
%         offset/.default=1}
         
%\usepackage{pgf-tikz}

\usepackage{pgfplots}
% this file ensures that all packages will be loaded so that TpX works properly.
% it is based on TpX's helpfile.

% comment unused packages below
\usepackage{color}
%\pdfoutput=0 % uncomment this to run PDFTeX in TeX mode
\usepackage{ifpdf}
\ifx\pdftexversion\undefined %if using TeX
  \usepackage{graphicx}
\else %if using PDFTeX
  \usepackage{graphicx} %-- kann nicht kompellieren, keine ahnung woran es liegt \usepackage[pdftex]{graphicx}
\fi

\ifpdf %if using PDFTeX in PDF mode
  \DeclareGraphicsExtensions{.pdf,.png,.mps}
  \usepackage{pgf}
  \usepackage{tikz}
\else %if using TeX or PDFTeX in TeX mode
  \usepackage{graphicx}
  \DeclareGraphicsExtensions{.eps,.bmp}
  \DeclareGraphicsRule{.emf}{bmp}{}{}% declare EMF filename extension
  \DeclareGraphicsRule{.png}{bmp}{}{}% declare PNG filename extension
  \usepackage{pgf}
  \usepackage{tikz}
  \usepackage{pstricks}%variant: \usepackage{pst-all}
\fi
\usepackage{epic,bez123}
\usepackage{floatflt}% package for floatingfigure environment
\usepackage{wrapfig}% package for wrapfigure environment

\usetikzlibrary{patterns,shapes,arrows}


%% Tabellensatz
\usepackage{array}
\usepackage{tabu}
%\usepackage{tabularx}      % Einige Tabellensatzerweiterungen
%\usepackage{booktabs}      % H�bsche Tabellen (unbedingt nutzen)
\usepackage{multirow}      % Mehrzeilige Zellen
\usepackage{supertabular}  % Tabell �ber mehrere Seiten
%\usepackage{caption}       % Tabellen�berschrift

%\usepackage{tabularx} %erstellt Zeilenumbruch in Tabellen bestimmter Gr��e: %\begin{tabularx}{\textwidth}{|c|c|X|c|} ...\end{tabularx}
\usepackage{longtable} % Seitenumbruch in Tabelle
\usepackage{booktabs} %F�r dicke Linien in Tabellen, z.B. \toprule, \midrule, \cmidrule(lr){4-5} oder \bottomrule

\usepackage[]{subfigure}
%\usepackage{subcaption}


\tikzstyle{decision} = [diamond, draw, fill=blue!20, 
    text width=4.5em, text badly centered, node distance=3cm, inner sep=0pt]
\tikzstyle{block} = [rectangle, draw, fill=blue!20, 
    text width=6em, text centered, rounded corners, minimum height=4em]
\tikzstyle{line} = [draw, -latex']
\tikzstyle{cloud} = [draw, ellipse,fill=red!20, node distance=3cm,
    minimum height=2em]
    
\errorcontextlines 10000
