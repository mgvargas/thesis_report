%*************************************************************
% Master Thesis                                            *
% Ing. Minerva Gabriela Vargas Gleason                       *
%  IAI - Institute of Artificial Intelligence                 *
% Universit�t Bremen                                         *
%                                                            *
% pdfLaTex                                                   *
% Editor: TeXnicCenter                                       *
%*************************************************************


%Seitenlayout f�r die Seiten auf denen ein neues Kapitel beginnt
%--> + Trennlinie + Name + Trennlinie
  
  
\appendix
\chapter{Appendix}\label{A1}
\markright{A. Appendix}
\vspace{-20pt}

\section{Github Repositories}\label{A2}

This project was developed in a computer running Ubuntu 16.04, no special hardware is required to execute the generated code.

However, the user needs to have some software and packages installed: 
\begin{itemize}
	\item \textbf{ROS:} With a configured workspace. The ROS distro used was \textit{kinetic}. Instalation instructions can be found in ROS.org\footnote{http://wiki.ros.org/ROS/Installation}
	\item \textbf{MoveIt!:} Used here the ROS kinetic release\footnote{http://moveit.ros.org/install/} for collision detection. Requires previous installation of ROS
\end{itemize}

Here you can find all the Github repositories with the code required to run this project.

\begin{itemize}
	\item Code generated during this project:
	
	\hspace{1cm}\texttt{\url{https://github.com/mgvargas/iai_markers_tracking}}
	
	\hspace{1cm}\texttt{\url{https://github.com/mgvargas/iai_trajectory_generation_boxy}}

	\hspace{1cm}\texttt{\url{https://github.com/mgvargas/qpOASES}}	
	\item Boxy's model and description:
	
	\hspace{1cm} \texttt{\url{https://github.com/code-iai/iai_robots}} 
	
	\item Robot's kinematic simulator:

	\hspace{1cm} \texttt{\url{https://github.com/code-iai/iai_naive_kinematics_sim}}
\end{itemize}

In the Readme file of the \texttt{iai\_trajectory\_generation\_boxy} package, you can find the instructions to install all the required repositories.

\section{Code Execution}

The \texttt{projection\_system.launch} is the main launch file, contained in the \texttt{iai\_trajectory\_\\generation\_boxy} package. Here, you can configure if the system will be executed on the robot or just as simulation. When the \texttt{sim} argument is \texttt{false} and the \texttt{boxy} is \texttt{true}, the system is configured to run on the Boxy robot receiving images from the Microsoft Kinect 2.

To configure the topics where the projection system subscribes to, you must change the corresponding parameters in the \texttt{markers\_detection.launch} file of the \texttt{iai\_markers\_tracking} package.

In order to send a command to the robot, you must send a request to the \texttt{ProjectedGrasping} action with the name of the object to grasp. This can be done simply by publishing the command to a ROS topic:

\begin{lstlisting}
rostopic pub /projected_grasping_server/goal iai_trajectory_generation_boxy/ProjectedGraspingActionGoal "header:
seq: 0
stamp:
secs: 0
nsecs: 0
frame_id: ''
goal_id:
stamp:
secs: 0
nsecs: 0
id: 'SOME_ID'
goal:
object: 'OBJECT_NAME'" 

\end{lstlisting}

The object must be already in the database for the robot to detect it and have the possible grasping poses available. 

New objects can be added any time to the \texttt{database.yaml} file of the \texttt{iai\_markers\_tracking} package following the same syntax of the other objects, explained in section \ref{sec:db}.

The file \texttt{send\_commands\_to\_boxy.py} contains the PI controller in charged of sending the obtained trajectory to the robot.

\todo{explain how to run the code. Required?}

