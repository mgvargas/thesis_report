%*************************************************************
% Master Project                                             *
% Ing. Minerva Gabriela Vargas Gleason                       *
% IAI - Institute of Artificial Intelligence                 *
% Universität Bremen                                         *
%                                                            *
% pdfLaTex                                                   *
% Editor: TeXnicCenter                                       *
%*************************************************************


\chapter{Discussion \& Conclusion}
The research of robots that are able to perform complex task in open environments, such as collaborative work with humans, is a growing topic in the research community. These robots need the capability of analysing and understanding the information perceived by their sensors in order to define the best curse of action to fulfill a given task.

Collaborative projects such as RoboHow and SAPHARI are working on improving the human-robot interaction. The main aspect addressed by these projects is obtaining a safe and intuitive interaction in collaborative tasks between humans and robots.

This is done with the main goal of making human-robot interaction a reality, using a combination of cognitive reaction and a human-friendly hardware and software design, allowing collaborative work in several areas, such as assembly lines and medical surgeries, as well as every-day-tasks interaction at home.

The results of this project can help researchers improve the autonomy of robots, by being able to change the orientation of their sensors as required.

Some remarks I want to point out after finishing this project are:
\begin{itemize}
	\item Virtual links can be used to easily solve planning problems when the orientation of a sensor is involved. A virtual link can simulate the normal vector that points out of the sensor and reaches the object/area the sensor is analysing. This can be applied, for example, to orient cameras and laser sensors
	\item MoveIt! is a powerful planning tool that allows you to select the planning algorithm that better adapts to your specific application. However, one must consider that configuring MoveIt! can take a long time if problems with the URDF are found or if the robot's drivers are not compatible with the software
	\item When solving an already existing problem, it is important to consider which solution will be better for the user. In this case, providing two methods of setting a goal for the UR3 brings more flexibility, so the program can be used in several applications
\end{itemize}

I think there are still some areas of opportunity in MoveIt!. Currently, the software uses two files to describe the kinematic and semantic properties of the robot, these files are the URDF and the SRDF. The SRDF can be seen as an extension of the URDF, it represents the semantic information of the robot and is closely linked to the URDF. Whenever the URDF changes, the SRDF must also be updated. Unifying both files in one with the complete representation of the robot could minimize errors while programming.